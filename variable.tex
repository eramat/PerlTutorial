\section{Les variables et les types}

\begin{frame}[fragile]
  \frametitle{Les variables}

  \begin{block}{Définition}
    \begin{itemize}
      \item Perl est un langage faiblement typé
      \item une variable est ``typée'' par sa valeur et son contexte
        d'utilisation
      \item le nom d'une variable est composé de lettres, de chiffres et du
        caractères \_, et ne peut pas commencer par un chiffre
      \item le caractère de ponctuation précédant le nom de la variable
        indique son type
    \end{itemize}
  \end{block}

  \begin{exampleblock}{Les types}
    \begin{itemize}
    \item les scalaires commencent par \$
    \item les tableaux par @
    \item les tables de hachage (ou dictionnaires) par \%
    \end{itemize}
  \end{exampleblock}

\end{frame}

\begin{frame}[fragile]
  \frametitle{Les variables}

  \begin{block}{Définition}
    \begin{itemize}
      \item il n'est pas nécessaire de déclarer les variables mais c'est
        préférable (cela permet de bien poser les variables utilisées)
      \item dès l'utilisation d'une variable, elle est déclarée
      \item par défaut, une variable de type scalaire contient la valeur
        \textit{undef}
      \item on déclare une variable de la manière suivante :
        \begin{lstlisting}[language=perl]
my $x;
my $x = 'Hello world';
        \end{lstlisting}
    \end{itemize}
  \end{block}

  \begin{alertblock}{Visibilité}
    \begin{itemize}
    \item une variable est accessible dans le bloc où elle est définie
    \item au sein d'un bloc, les variables peuvent être définies où l'on veut
    \end{itemize}
  \end{alertblock}

\end{frame}

\begin{frame}[fragile]
  \frametitle{Les types}
  \framesubtitle{Les scalaires}

  \begin{block}{Définition}
    \begin{itemize}
    \item un scalaire est une simple valeur
    \item par exemple; les nombres, les chaînes de caractères, \ldots
    \end{itemize}
  \end{block}

  \begin{exampleblock}{Les nombres}
    \begin{itemize}
    \item les nombres sont représentés en double précision
    \item un nombre peut être un entier ou un réel
    \item quelques exemples de valeurs possibles :
      \begin{itemize}
      \item 4294967295
      \item 4\_294\_967\_295
      \item 0xFFFFFFFF
      \item 0xFF\_FF\_FF\_FF
      \item 0b1111\_1111\_1111\_1111
      \end{itemize}
    \item le caractère \_ permet de rendre lisible certaines valeurs et il est
      ignoré
    \end{itemize}
  \end{exampleblock}

\end{frame}

\begin{frame}[fragile]
  \frametitle{Les types}
  \framesubtitle{Les scalaires}

  \begin{block}{Les opérateurs sur les nombres}
    \begin{itemize}
    \item arithmétique : +, -, * et /
    \item exponentiel : **
    \item modulo : %
    \item comparaison : <, >, ==, !=, <=, >=
    \item logique : not ou !, \&\&, ||
    \end{itemize}
  \end{block}

\end{frame}

\begin{frame}[fragile]
  \frametitle{Les types}
  \framesubtitle{Les scalaires}

  \begin{block}{Les chaînes de caractères}
    \begin{itemize}
    \item deux délimiteurs sont possibles : double quote (") et simple quote(')
    \item dans le cas de la simple quote : aucune interprétation n'est faite
    \item par exemple, \\n est vu comme une chaîne à 2 caractères et non
      comme un retour chariot
    \item dans le cas de la double quote, les séquences spéciales sont
      interprétées
      \begin{lstlisting}[language=perl]
"hello ${name}\\n" ou "hello $name\\n"
      \end{lstlisting}
    \item \$name est remplacé par la valeur de la variable
    \end{itemize}
  \end{block}

  \begin{exampleblock}{Opérateurs}
    \begin{itemize}
    \item concaténation avec le point
      \begin{lstlisting}[language=perl]
'Hello ' . 'World!' # 'Hello World!'
      \end{lstlisting}
    \item réplication avec x
      \begin{lstlisting}[language=perl]
'Hello ' x 3   # 'Hello Hello Hello '
      \end{lstlisting}
    \end{itemize}
  \end{exampleblock}

\end{frame}

\begin{frame}[fragile]
  \frametitle{Les types}
  \framesubtitle{Les scalaires}

  \begin{block}{Les opérateurs de comparaison}
    \begin{itemize}
    \item les noms des opérateurs est comme en Fortran
    \item c'est une comparaison basée sur les codes ASCII
    \end{itemize}
  \end{block}

  \begin{exampleblock}{Liste}
    \begin{lstlisting}{language=perl}
'Peter' gt 'Joey'     # greater than, TRUE
'Sammy' lt 'Dean'     # less than, FALSE
'Frank' eq 'frank'    # equals, FALSE
'Frank' ne 'Peter'    # not equals, TRUE
'Frank' ge 'Dean'     # greater or equal, TRUE
'Frank' le 'Joey'     # lesser or equal, TRUE
'2' gt '10'           # TRUE
    \end{lstlisting}
  \end{exampleblock}

\end{frame}

\begin{frame}[fragile]
  \frametitle{Les types}
  \framesubtitle{Les scalaires}

  \begin{block}{Le contexte}
    En fonction des opérateurs et des opérandes, un scalaire peut
    être vu comme un nombre ou une chaîne de caractères
  \end{block}

  \begin{exampleblock}{Exemple}
    \begin{lstlisting}{language=perl}
"1234 My Way" => 1234

5 + "1234 My Way" => 1239

'$' . (102/5) => '$20.4'
    \end{lstlisting}
  \end{exampleblock}

\end{frame}

\begin{frame}[fragile]
  \frametitle{Les types}
  \framesubtitle{Les listes}

  \begin{block}{Définition}
    \begin{itemize}
      \item une liste est un ensemble de scalaires
      \item les éléments sont indexés et les indices commencent par 0
      \item les paranthèses marquent la définition d'une liste
        \begin{lstlisting}{language=perl}
('Lapin', 'Poule', 'Canard')
        \end{lstlisting}
      \item il est possible de sélectionner une sous-partie de la liste
        \begin{lstlisting}{language=perl}
('Lun', 'Mar', 'Mer', 'Jeu', 'Ven', 'Sam', 'Dim')[1,5]
('Lun', 'Mar', 'Mer', 'Jeu', 'Ven', 'Sam', 'Dim')[-2,-1]
        \end{lstlisting}
    \end{itemize}
  \end{block}

  \begin{alertblock}{Liste et fonction}
    Certaines fonctions retournent des listes
    \begin{lstlisting}{language=perl}
($sec,$min,$hour,$mday,$mon,$year,$wday,$yday,$isdst) = localtime();
($mon, $mday, $year) = (localtime())[4,3,5]
    \end{lstlisting}
  \end{alertblock}

\end{frame}

\begin{frame}[fragile]
  \frametitle{Les types}
  \framesubtitle{Les listes}

  \begin{block}{Accès aux éléments d'une liste}
    \begin{itemize}
    \item il faut utiliser \$ pour accéder à un seul élément d'une liste
      contenant des scalaires
      \begin{lstlisting}{language=perl}
print $fruits[0];
print $fruits[2];
      \end{lstlisting}
    \item si on veut accèder à plusieurs éléments alors il faut utiliser \@
      \begin{lstlisting}{language=perl}
print @fruits[1 .. 3];
      \end{lstlisting}
    \end{itemize}
  \end{block}

\end{frame}

\begin{frame}[fragile]
  \frametitle{Les types}
  \framesubtitle{Les listes}

  \begin{exampleblock}{Affectation}
    \begin{itemize}
    \item affectation d'une case de la liste
      \begin{lstlisting}{language=perl}
$fruits[4] = 'pomme';
$fruits[2] = 'poire';
      \end{lstlisting}
    \item affectation d'une sous-partie de la liste
      \begin{lstlisting}{language=perl}
@array[4, 7 .. 9] = ('quatre', 'sept', 'huit', 'neuf');
      \end{lstlisting}
    \end{itemize}
  \end{exampleblock}

  \begin{alertblock}{Permutation}
    Permutation des éléments d'une liste (\textit{swap})
    \begin{lstlisting}{language=perl}
@array[1, 2] = @array[2, 1];
($x, $y) = ($y, $x);
    \end{lstlisting}
  \end{alertblock}

\end{frame}

\begin{frame}[fragile]
  \frametitle{Les types}
  \framesubtitle{Les listes}

  \begin{block}{Taille d'une liste}
    \begin{itemize}
    \item \textit{\$\#array} est l'index du dernier élément de la liste
      \textit{\@array}
    \item le nombre d'éléments est donc : \textit{\$\#array + 1}
    \item de manière plus simple, il suffit d'affecter
      la liste à un scalaire : \textit{\$count = \@array;}
    \end{itemize}
  \end{block}

\end{frame}

\begin{frame}[fragile]
  \frametitle{Les types}
  \framesubtitle{Les listes}

  \begin{exampleblock}{Quelques opérateurs}
    \begin{itemize}
    \item l'opérateur \textit{shift} supprime le premier élément d'une liste
      \begin{lstlisting}{language=perl}
@num = 4 .. 7;
$first = shift @num;    # @num is ( 5..7 )
      \end{lstlisting}
    \item l'opérateur \textit{unshift} ajoute un élément en début de liste
      \begin{lstlisting}{language=perl}
unshift @num, $first; # @num is ( 4 .. 7 )
unshift @num, 1 .. 3; # @num is ( 1 .. 7 )
      \end{lstlisting}
    \item les opérateurs \textit{push} et \textit{pop} permettent de faire la
      même chose
      \begin{lstlisting}{language=perl}
$last = pop @num;        # @num is ( 1 .. 6 )
push @num, $last, 8, 9;  # @num is ( 1 .. 9 )
      \end{lstlisting}
    \end{itemize}
  \end{exampleblock}

\end{frame}

\begin{frame}[fragile]
  \frametitle{Les types}
  \framesubtitle{Les dictionnaires}

  \begin{block}{Définition}
    \begin{itemize}
    \item un dictionnaire est une table associant une clé à une valeur
    \item deux syntaxes pour l'initialisation:
      \begin{itemize}
      \item à l'aide d'une liste
        \begin{lstlisting}{language=perl}
%french = ('one', 'un', 'two', 'deux', 'three', 'trois');
        \end{lstlisting}
      \item à l'aide de l'opérateur \textit{=>}
        \begin{lstlisting}{language=perl}
%german = (one   => 'ein',
           two   => 'zwei',
           three => 'drei');
        \end{lstlisting}
      \end{itemize}
    \end{itemize}
  \end{block}

  \begin{exampleblock}{Accès}
    Comme pour les listes, l'accès à une valeur utilise le symbole \$
    \begin{lstlisting}{language=perl}
$three = $french{three};
    \end{lstlisting}
    et l'accès à plusieurs valeurs utilise le symbole \@
    \begin{lstlisting}{language=perl}
@mots = @french{'one','two','three'};
    \end{lstlisting}
  \end{exampleblock}

\end{frame}

\begin{frame}[fragile]
  \frametitle{Les types}
  \framesubtitle{Les dictionnaires}

  \begin{block}{Affectation}
    On peut affecter :
    \begin{itemize}
    \item une seule entrée du dictionnaire
      \begin{lstlisting}{language=perl}
$french{four} = 'quatre';
$french{five} = 'cinq;
      \end{lstlisting}
    \item un ensemble d'entrées
      \begin{lstlisting}{language=perl}
@french{'six', 'seven'} = ('six', 'sept');
      \end{lstlisting}
    \item échange d'entrées
      \begin{lstlisting}{language=perl}
@hash{'foo', 'bar'} = @hash{'bar', 'foo'};
      \end{lstlisting}
    \end{itemize}
  \end{block}

\end{frame}

\begin{frame}[fragile]
  \frametitle{Les types}
  \framesubtitle{Les dictionnaires}

  \begin{block}{Tests d'existence}
    Deux tests possibles :
    \begin{itemize}
    \item sur les clés
      \begin{lstlisting}{language=perl}
$ok = exists $french{'ten'};
      \end{lstlisting}
    \item sur les valeurs
      \begin{lstlisting}{language=perl}
$ok = defined $french{'dix'};
      \end{lstlisting}
    \end{itemize}
  \end{block}

  \begin{alertblock}{Attention !}
    Si la clé n'existe pas, la variable \$ok n'existe pas.
  \end{alertblock}

\end{frame}

\begin{frame}[fragile]
  \frametitle{Les types}
  \framesubtitle{Les dictionnaires}

  \begin{block}{Quelques opérateurs}
    \begin{itemize}
    \item \textit{keys} : récupére la liste de toutes les clés (dans un ordre
      quelconque)
      \begin{lstlisting}{language=perl}
@keys = keys %french;
      \end{lstlisting}
    \item \textit{values} : récupére la liste de toutes les valeurs
      \begin{lstlisting}{language=perl}
@values = values %french;
      \end{lstlisting}
    \item \textit{each} : récupére la prochaine pair (cké, valeur)
      \begin{lstlisting}{language=perl}
($key, $value) = each %french;
      \end{lstlisting}
    \end{itemize}
  \end{block}

\end{frame}

\begin{frame}[fragile]
  \frametitle{Les variables spéciales}

  \begin{block}{Définition}
    Les variables spéciales :
    \begin{itemize}
    \item sont en majuscules ou utilisent des symboles de ponctuation comme nom
    \item sont des variables par défaut (par exemple, l'élément parcouru dans
      une boucle, la ligne courante lue dans un fichier, \ldots)
    \end{itemize}
  \end{block}

  \begin{exampleblock}{Exemples}
    \begin{itemize}
    \item \textit{\$\_} est la variable par défaut (ou la variable contenant
      la "valeur courante" ; par exemple, l'élément parcouru dans une boucle)
    \item \textit{\$.} est le numéro de la ligne courante (lors d'un parcours
      d'un fichier)
    \item \textit{\$,} est le séparateur de champs pour la fonction print()
    \item \textit{@\_} contient les arguments passés au sous-programme en cours
    \end{itemize}
  \end{exampleblock}

\end{frame}

\begin{frame}[fragile]
  \frametitle{Les variables spéciales}

  \begin{exampleblock}{Encore des exemples}
    \begin{itemize}
    \item \textit{@ARGV} contient la liste des paramètres passés sur la
      ligne de commande au script en cours d'exécution
    \item \textit{\%ENV} ontient l'ensemble des variables d'environnement
      définies lorsque le script a été lancé
    \end{itemize}
  \end{exampleblock}

\end{frame}

\begin{frame}[fragile]
  \frametitle{Le contexte}

  \begin{alertblock}{Résumé}
    \begin{center}
      \rowcolors[]{1}{blue!5}{blue!8}
      \begin{tabular}{l!{\vrule}cc}
        \rowcolor{blue!20}& \textbf{scalaire} & \textbf{liste}\\\hline
        \texttt{\$a}      & un scalaire           & une liste à un élément \\
        \texttt{@a}       & le nombre d'éléments  & un tableau \\
        \texttt{\$a[\$n]} & un élément du tableau & une liste à un élément \\
        \texttt{@a[@n]}   & le dernier élément de la sélection & une sélection du tableau \\
        \texttt{\%a}      & les stats du dictionnaire & la liste des pairs (clé,valeur) \\
        \texttt{\$a{\$n}} & un élément du dictionnaire & une liste à un élément\\
        \texttt{@a{@n}}   & le dernier élément de la sélection & une sélection du dictionnaire \\
      \end{tabular}
    \end{center}
  \end{alertblock}

\end{frame}

