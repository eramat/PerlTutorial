\section{Les structures de contrôle}

\begin{frame}[fragile]
  \frametitle{Les structures de contrôle}

  \begin{block}{Les blocs}
    \begin{itemize}
    \item un bloc d'instructions est défini entre accolades
    \item en général, un bloc est précédé par une instruction de contrôle
    \item si l'accolade ouvrante (marquant le début du bloc) n'est pas
      précédée par une instruction de contrôle (par exemple, un \textit{if})
      alors ce bloc n'a aucun effet
    \end{itemize}
  \end{block}

  \begin{exampleblock}{Un bloc simple}
    \begin{lstlisting}{language=perl}
{
  instruction_1;
  instruction_2;
  ...
  instruction_n;
}
    \end{lstlisting}
  \end{exampleblock}

\end{frame}

\begin{frame}[fragile]
  \frametitle{Les structures de contrôle}

  \begin{block}{Le bloc \textit{if}}
    \begin{itemize}
    \item le bloc est exécuté si la condition est vraie
    \item un bloc \textit{else} peut être ajouté si on veut faire exécuter
      des instructions dans le cas où la condition est fausse
    \item l'instruction \textit{elsif} permet de construire des conditions
      en cascade
    \end{itemize}
  \end{block}

  \begin{alertblock}{Syntaxe}
    \begin{lstlisting}{language=perl}
if( $condition )
{
  # si $condition est vraie
}
elsif( $une_autre_condition )
{
  # si $une_autre_condition est vraie
}
else
{
  # sinon
}
    \end{lstlisting}
  \end{alertblock}

\end{frame}

\begin{frame}[fragile]
  \frametitle{Les structures de contrôle}

  \begin{block}{Le bloc \textit{unless}}
    \begin{itemize}
    \item \textit{unless} est l'inverse de \textit{if}, le bloc est exécuté si
      la condition est fausse
    \item un bloc \textit{else} peut être ajouté si on veut faire exécuter
      des instructions dans le cas où la condition est vraie
    \item l'instruction \textit{elsif} permet de construire des conditions
      en cascade même si la première condition est un \textit{unless}
    \end{itemize}
  \end{block}

  \begin{alertblock}{Syntaxe}
    \begin{lstlisting}{language=perl}
unless( $condition )
{
  # si $condition est fausse
}
    \end{lstlisting}
  \end{alertblock}

\end{frame}

\begin{frame}[fragile]
  \frametitle{Les structures de contrôle}

  \begin{block}{Les modificateurs}
    \begin{itemize}
    \item une instruction simple peut être suivie d'une condition (\textit{if}
      ou \textit{unless}, appelé modificateur
    \item si la condition du modificateur est vraie alors l'instructeur
      est réalisée
    \item on ne peut pas mettre plusieurs modificateurs sur une même
      instruction
    \end{itemize}
  \end{block}

  \begin{exampleblock}{Exemple}
    \begin{lstlisting}{language=perl}
$message = 'no' if $total == 17;
$message = 'yes' unless $total >= 17;
print "Le total est $total\n" if $debug;
    \end{lstlisting}
  \end{exampleblock}

\end{frame}

\begin{frame}[fragile]
  \frametitle{Les structures de contrôle}

  \begin{block}{\textit{while} / \textit{until}}
    \begin{itemize}
    \item la première forme de boucle est basée sur un simple test
    \item deux possibilités :
      \begin{itemize}
      \item \texttt{tant que} (\textit{while}) : le bloc est répété tant que
        la condition est vraie
      \item \texttt{jusqu'à ce que} (\textit{until}) : le bloc est répéré
        jusqu'à ce que la condition soit vraie (donc répété tant que la
        condition est fausse)
      \end{itemize}
    \end{itemize}
  \end{block}

  \begin{exampleblock}{Syntaxe}
    \begin{lstlisting}{language=perl}
while( $condition )
{ ... }
    \end{lstlisting}
    \begin{lstlisting}{language=perl}
until( $condition )
{ ... }
    \end{lstlisting}
  \end{exampleblock}

  \begin{alertblock}{Attention !}
    La condition est évaluée au moins une fois !
  \end{alertblock}

\end{frame}

\begin{frame}[fragile]
  \frametitle{Les structures de contrôle}

  \begin{block}{La boucle \textit{for}}
    \begin{itemize}
    \item une boucle \textit{for} est composée de 4 éléments :
      \begin{itemize}
      \item la phase d'initialisation (en général, une variable mise à zéro)
      \item le test de poursuite de l'itération
      \item la phase d'incrémentation
      \item le bloc d'instructions à répéter
      \end{itemize}
    \end{itemize}
  \end{block}

  \begin{exampleblock}{Exemple}
    \begin{lstlisting}{language=perl}
for( $i = 0; $i < 21; $i++)
{
  print $i;
}
    \end{lstlisting}
  \end{exampleblock}

  \begin{alertblock}{Boucle infinie}
    L'un ou tous les éléments du \textit{for} peuvent être omis.
    \begin{lstlisting}{language=perl}
for(;;) { ... }
    \end{lstlisting}
  \end{alertblock}

\end{frame}

\begin{frame}[fragile]
  \frametitle{Les structures de contrôle}

  \begin{block}{\textit{foreach}}
    \begin{itemize}
    \item dans le cas de l'itération d'une liste, il est possible d'utiliser
      une instruction d'itération spéciale : \textit{foreach}
    \item les éléments parcourus sont référencés par la variable spéciale
      \$\_
    \end{itemize}
  \end{block}

  \begin{exampleblock}{Exemple}
    \begin{lstlisting}{language=perl}
foreach( @users )
{
  print "$_\n est actif";
}

foreach $user ( @users )
{
  print "$user est actif\n";
}
    \end{lstlisting}
  \end{exampleblock}
%$
\end{frame}

\begin{frame}[fragile]
  \frametitle{Les structures de contrôle}

  \begin{block}{\textit{last}}
    \begin{itemize}
    \item par défaut, une boucle s'arrête lorsque la condition d'arrêt est vraie
      ou fausse selon le type de boucle (sauf dans le cas de la boucle infinie)
    \item l'instruction \textit{last} mets fin à une boucle (équivalent du
      \textit{break} du C)
    \end{itemize}
  \end{block}

  \begin{exampleblock}{Exemple}
    \begin{lstlisting}{language=perl}
while( $condition )
{
  ...
  last if $une_autre_condition
}

foreach ( @users )
{
  last if $_ eq 'root';
  print "$_ est actif\n";
}
    \end{lstlisting}
  \end{exampleblock}

\end{frame}

\begin{frame}[fragile]
  \frametitle{Les structures de contrôle}

  \begin{block}{\textit{next}}
    \begin{itemize}
    \item par défaut, l'ensemble des instructions du bloc d'une boucle sont
      exécutées
    \item l'instruction \textit{next} permet de passer à l'itération
      suivante sans exécuter les instructions suivantes (équivalent du
      \textit{continue} du C)
    \end{itemize}
  \end{block}

  \begin{exampleblock}{Exemple}
    \begin{lstlisting}{language=perl}
while( $condition )
{
  ...
  next unless $x > 0;
  ...
}

foreach ( @users )
{
  next unless $_ ne 'root';
  print "$_ est actif\n";
}
    \end{lstlisting}
  \end{exampleblock}

\end{frame}

\begin{frame}[fragile]
  \frametitle{Les structures de contrôle}

  \begin{block}{\textit{redo}}
    \begin{itemize}
    \item par défaut, l'ensemble des instructions du bloc d'une boucle sont
      exécutées une et une fois par itération
    \item l'instruction \textit{redo} permet de rééxécuter les instructions
      d'une itération (sans réévaluation de la condition et/ou de
      l'incrémentation)
    \end{itemize}
  \end{block}

  \begin{exampleblock}{Exemple}
    \begin{lstlisting}{language=perl}
while( $condition )
{
  ...
  redo if $une_autre_condition
}
    \end{lstlisting}
  \end{exampleblock}

  \begin{alertblock}{\textit{redo} dans un bloc "nu"}
    Tant que la condition est vraie, le bloc est redémarré
    \begin{lstlisting}{language=perl}
{
...
redo unless $condition;
}
    \end{lstlisting}
  \end{alertblock}
%$
\end{frame}

\begin{frame}[fragile]
  \frametitle{Les structures de contrôle}

  \begin{block}{Les blocs labelissés}
    \begin{itemize}
    \item un bloc peut être nommé à l'aide d'un label (une étiquette)
    \item les labels sont utilisables au sein des instructions \textit{next},
      \textit{last} et \textit{redo} dans le cas des blocs imbriqués
    \item on peut alors spécifier le bloc que l'on redémarre (dans le cas du
      \textit{next} même si l'instruction \textit{next} est exécuté dans un bloc
      qui est lui-même dans le bloc à redémarrer
    \end{itemize}
  \end{block}

  \begin{exampleblock}{Exemple}
    Si la deuxième condition est vraie alors la boucle \textit{LOOP\_2} est
    stoppée et la boucle \textit{LOOP\_1} passe à l'itération suivante
    \begin{lstlisting}{language=perl}
LOOP_1: foreach ( @users )
{
  ...
  LOOP_2: while( $condition )
  {
    ...
    next LOOP_1 if $une_autre_condition;
  }
}
    \end{lstlisting}
  \end{exampleblock}

\end{frame}

