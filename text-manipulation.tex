\section{Manipulation de texte}

\begin{frame}[fragile]
  \frametitle{Manipulation de texte}

  \begin{block}{Définition}
    Les scripts Perl font principalement du traitement sur des fichiers
    et du texte :
    \begin{itemize}
    \item recherche de chaînes de caractères
    \item extraction, découpage, substitution, remplacement, ...
    \item transformation
    \end{itemize}
  \end{block}

\end{frame}

\begin{frame}[fragile]
  \frametitle{Manipulation de texte}

  \begin{block}{Recherche d'une sous-chaîne}
    \begin{itemize}
    \item la fonction \textit{index} recherche une sous-chaîne dans une
      chaîne
    \item elle retourne la position du premier caractère de la sous-chaîne
      retrouvée dans la chaîne
    \item cette position est égale à -1 si la sous-chaîne n'est pas trouvée
    \item on peut aussi démarrer la recherche à partir d'une certaine position
    \end{itemize}
  \end{block}

  \begin{exampleblock}{Exemples}
    \begin{lstlisting}[language=perl]
$pos = index($string, $substring);

$pos = index("Hello world", "Hello");  # $pos = 0
$pos = index("Hello world", "world");  # $pos = 6
$pos = index("Hello world", "toto");   # $pos = -1

$pos = index("Hello world", "o", 6);  # on debute la recherche a 6

$pos = rindex("Hello world", "o"); # recherche en partant de la fin
    \end{lstlisting}
  \end{exampleblock}
%$
\end{frame}

\begin{frame}[fragile]
  \frametitle{Manipulation de texte}

  \begin{block}{Extraction d'une sous-chaîne}
    \begin{itemize}
    \item la fonction \textit{substr} récupère une sous-chaîne d'une certaine
      longueur dans une chaîne
    \item elle peut aussi permettre le remplacer d'une sous-chaîne par une
      autre sous-chaîne en la combinant avec l'opérateur d'affectation
    \end{itemize}
  \end{block}

  \begin{exampleblock}{Exemples}
    \begin{lstlisting}[language=perl]
$sub_str = substr $string, $start, $length

$str = substr "Hello world", 6, 2; # $str = "wo"
$str = substr "Hello world", 6     # $str = "world"

$name = 'Hello world';
$str = substr $name, index($name, 'w'); # $str = "world"

substr($name,0,5) = 'Good ';        # "Good  world"
substr($name,0,5) =~ s/Hello/Good/;
    \end{lstlisting}
  \end{exampleblock}
%$
\end{frame}

\begin{frame}[fragile]
  \frametitle{Manipulation de texte}

  \begin{block}{Remplacement}
    \begin{itemize}
    \item la fonction \textit{tr} remplace des caractères par d'autres
      au sein du contenu de la variable \$\_
    \item la fonction retourne le nombre de caractères remplacés
    \item avec l'opérateur =\textasciitilde, on peut affecter le résultat à une
      autre variable que le variable spéciale \$\_
    \end{itemize}
  \end{block}

  \begin{exampleblock}{Exemples}
    \begin{lstlisting}[language=perl]
$_ = "hello world";
tr/eo/oe/;           # la lettre e est remplacee par le o et le o par e
tr/a-z/n-za-m/;      # le a est remplacé par n, b par o, ...
$v =~ tr/a-z/n-za-m/; # affectation d'une autre variable
    \end{lstlisting}
  \end{exampleblock}
%$
\end{frame}

\begin{frame}[fragile]
  \frametitle{Manipulation de texte}

  \begin{block}{\textit{split}}
    \begin{itemize}
    \item le traitement du texte provenant de fichiers de log conduit
      souvent à filtrer et à découper des chaînes
    \item la fonction \textit{split} permet de construire des listes de
      sous-chaînes à partir d'une chaîne selon certains critères :
      \begin{itemize}
      \item une expression régulière
      \item un délimiteur
      \end{itemize}
    \end{itemize}
  \end{block}

  \begin{exampleblock}{Exemples}
    \begin{itemize}
    \item selon une expression régulière
      \begin{lstlisting}[language=perl]
@names = split(/REGEX/, $string);
      \end{lstlisting}
    \item selon le séparateur :
      \begin{lstlisting}[language=perl]
@names = split(/:/, 'Toto:Titi:Tutu');       # @names = ["Toto", "Titi", "Tutu"]
      \end{lstlisting}
    \end{itemize}
  \end{exampleblock}
%$
\end{frame}

\begin{frame}[fragile]
  \frametitle{Manipulation de texte}

  \begin{exampleblock}{Exemples}
    \begin{itemize}
    \item selon le séparateur : et des champs vides
      \begin{lstlisting}[language=perl]
@names = split(/:/, ':::Toto:Titi:Titi:::');  # le tabeau va contenir des cases vides
      \end{lstlisting}
    \item par défaut
      \begin{lstlisting}[language=perl]
@name = split();       # split le contenu de $_ selon le separateur espace
      \end{lstlisting}
    \end{itemize}
  \end{exampleblock}
%$

  \begin{alertblock}{\textit{join}}
    La fonction \textit{join} réalise l'inverse de \textit{split} : à
    partir d'une liste de chaînes de caractères, on construit une chaîne où
    les éléments sont séparés par un séparateur
    \begin{lstlisting}[language=perl]
$str = join($separator, @bits);

@names = split(/:/, "Toto:Titi:Tutu");
$str = join(':', @names);      # reconstruction de la meme chaine
    \end{lstlisting}
  \end{alertblock}
%$
\end{frame}

\begin{frame}[fragile]
  \frametitle{Manipulation de texte}

  \begin{block}{Quelques fonctions supplémentaires}
    \begin{itemize}
    \item \textit{chop} supprime le dernier caractère d'une chaîne
    \item \textit{chomp} supprime le dernier caractère d'une chaîne si c'est
      un retour chariot (\textbackslash n)
    \item \textit{lc} transforme tout en minuscule
    \item \textit{lcfirst} transforme seulement le premier caractère en
      minuscule
    \item \textit{uc} transforme tout en majuscule
    \item \textit{ucfirst} transforme seulement le premier caractère en
      majuscule
    \end{itemize}
  \end{block}

\end{frame}
