\section{Les sous-routines}

\begin{frame}[fragile]
  \frametitle{Les sous-routines}

  \begin{block}{Définition}
    \begin{itemize}
    \item les sous-routines sont des fonctions
    \item elles peuvent être appelées avec ou sans paramètre
    \item elles sont définies n'importe où dans un programme
    \item elles sont nommées comme les variables
    \item le symbole \& est utilisé pour les appels
    \item la valeur retournée par la sous-routine est le résultat de la
      dernière expression évaluée (il n'y a donc pas de mot-clé spéficique)
    \end{itemize}
  \end{block}

  \begin{exampleblock}{Un exemple}
    \begin{lstlisting}[language=perl]
$result = &hello_world;   # appel à la sous-routine hello_world

sub hello_world
{
  "hello"."world";
}
    \end{lstlisting}
  \end{exampleblock}
%$
\end{frame}

\begin{frame}[fragile]
  \frametitle{Les sous-routines}

  \begin{alertblock}{Un autre exemple}
    \begin{lstlisting}[language=perl]
sub max
{
  if ( $x <= $y )
  {
    $y;
  }
  else
  {
    $x;
  }
}
    \end{lstlisting}
  \end{alertblock}
%$
  \begin{block}{\textit{return}}
    On peut aussi utiliser \textit{return}
    \begin{lstlisting}[language=perl]
sub max
{
  return $y if $x <= $y;
  $x;
}
    \end{lstlisting}
  \end{block}

\end{frame}

\begin{frame}[fragile]
  \frametitle{Les sous-routines}

  \begin{block}{Paramètres}
    \begin{itemize}
    \item on peut passer des paramètres aux sous-routinex lors de l'appel
    \item dans les sous-routines, les paramètres sont accessibles via la
      variable spéciale @\_ (une liste)
    \item chaque sous-routine a son propre @\_
    \end{itemize}
  \end{block}

  \begin{exampleblock}{Un exemple}
    \begin{lstlisting}[language=perl]
$result = &add($x, $y);

sub add
{
  ($m, $n) = @_;
  $m + $n;
}
    \end{lstlisting}
  \end{exampleblock}
%$
\end{frame}

\begin{frame}[fragile]
  \frametitle{Les sous-routines}

  \begin{block}{Variables locales}
    \begin{itemize}
    \item par défaut, toutes les variables sont globales
    \item on peut définir des variables locales de deux manières différentes :
      \begin{itemize}
      \item grâce à une déclaration via \textit{my}
      \item grâce à la fonction \textit{local}
      \end{itemize}
    \item dans le cas de la fonction \textit{local}, s'il existe une variable
      globale de même nom qu'une variable locale, la variable globale est cachée
      temporairement
    \item une variable locale peut être de n'importe quel type
    \end{itemize}
  \end{block}

\end{frame}

\begin{frame}[fragile]
  \frametitle{Les sous-routines}

  \begin{exampleblock}{Un exemple avec \textit{local}}
    \begin{lstlisting}[language=perl]
sub foo
{
  local($x, $y);
  ($x,$y) = @_;

  local ($x, $y) = @_;  # on peut assigner lors de la declaration

  ...
}
    \end{lstlisting}
  \end{exampleblock}

  \begin{alertblock}{Un exemple avec \textit{my}}
    \begin{lstlisting}[language=perl]
sub foo
{
  my ($x, $y) = @_;
  ...
}
    \end{lstlisting}
  \end{alertblock}

\end{frame}
