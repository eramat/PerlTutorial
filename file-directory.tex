\section{Les fichiers et les répertoires}

\begin{frame}[fragile]
  \frametitle{Les fichiers et les répertoires}

  \begin{block}{Définition}
    Il existe deux solutions pour interagir avec le système de fichiers :
    \begin{itemize}
    \item à l'aide des commandes système et la fonction \textit{exec} ou
      \textit{system} ou les expressions entre \textit{back-quote}
    \item à l'aide de fonctions Perl
    \end{itemize}
  \end{block}

  \begin{exampleblock}{Exemples sous Linux}
    Le changement de permissions sur un fichier:
    \begin{itemize}
    \item via \textit{exec} :
      \begin{lstlisting}{language=perl}
exec("chmod +x test.pl");
      \end{lstlisting}
    \item via la fonction \textit{chmod}
      \begin{lstlisting}{language=perl}
chmod(0666, @file_list);
      \end{lstlisting}
    \item via la fonction \textit{chmod} du module \textit{File}
      \begin{lstlisting}{language=perl}
use File::chmod;
chmod("+x", @files);
      \end{lstlisting}
    \end{itemize}
  \end{exampleblock}

\end{frame}
