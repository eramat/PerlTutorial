\section{Entrées et sorties}

\begin{frame}[fragile]
  \frametitle{Entrées et sorties}

  \begin{block}{Définition}
    \begin{itemize}
    \item le principal objectif d'un script est de faire des entrée-sorties
    \item il existe plusieurs méthodes :
      \begin{itemize}
      \item via les entrée-sorties standards (STDIN et STDOUT)
      \item via des fichiers
      \item \ldots
      \end{itemize}
    \end{itemize}
  \end{block}

  \begin{exampleblock}{Sortie standard}
    \begin{itemize}
    \item envoi de données vers la sortie standard via la fonction
      \textit{print}
      \begin{lstlisting}{language=perl}
print "Hello world!\n";
print STDOUT "Hello world!\n"
      \end{lstlisting}
    \item sortie à l'aide de la variable \$\_
      \begin{lstlisting}{language=perl}
print;
print STDOUT;
      \end{lstlisting}
    \item sortie de plusieurs éléments via un seul appel à \textit{print}
      \begin{lstlisting}{language=perl}
print "Hello", " world", "!\n";
print "Le nombre est : ", 10 + 10 + 1, "!\n";
      \end{lstlisting}
    \end{itemize}
  \end{exampleblock}

\end{frame}

\begin{frame}[fragile]
  \frametitle{Entrées et sorties}

  \begin{block}{Sorties formatées}
    \begin{itemize}
    \item \textit{printf} est un \textit{print} avec une chaîne de caractères
      de formatage
      \begin{lstlisting}{language=perl}
printf "Le nombre est %d!\n", 10 + 11;
printf "%s appartient au groupe %s\n", $user, $group;
      \end{lstlisting}
    \item la notion du format est identique à celui du langage C
      \begin{lstlisting}
printf "La taille du fichier est \$%.2f Mo\n", $size;
      \end{lstlisting}
    \item le mécanisme fonctionne aussi pour construire des chaînes de
      caractères
      \begin{lstlisting}
$str = sprintf "La taille du fichier est \$%.2f Mo\n", $size;
      \end{lstlisting}
    \end{itemize}
  \end{block}

\end{frame}

\begin{frame}[fragile]
  \frametitle{Entrées et sorties}

  \begin{block}{Entrée standard}
    \begin{itemize}
    \item les éléments de l'entrée standard (clavier ou via l'opérateur $<$)
      sont accessibles via \textit{<STDIN>}
      \begin{lstlisting}{language=perl}
print "Entrer votre login :";
$name = <STDIN>;
      \end{lstlisting}
    \item $<$ \ldots $>$ est l'opérateur de lecture d'un flux en général
    \item dans un contexte de liste, \textit{<STDIN>} retourne toutes les lignes
      en une seule fois
      \begin{lstlisting}{language=perl}
@lines = <STDIN>;
      \end{lstlisting}
    \end{itemize}
  \end{block}

  \begin{alertblock}{Suppression des sauts de ligne}
    La fonction \textit{chomp} retire les sauts de ligne d'une variable de type
    string
    \begin{lstlisting}{language=perl}
chomp($name = <STDIN>);
    \end{lstlisting}
  \end{alertblock}

\end{frame}

\begin{frame}[fragile]
  \frametitle{Entrées et sorties}

  \begin{block}{Boucle sur des entrées}
    \begin{itemize}
    \item les éléments mis sur l'entrée standard peuvent être lu dans une boucle
    \item la variable \$\_ peut être utilisée
    \begin{lstlisting}{language=perl}
while( <STDIN> )
{
  print "echo> $_";
}
    \end{lstlisting}
  \item sinon il faut définir une variable
    \begin{lstlisting}{language=perl}
while( defined($line = <STDIN>) )
{
  print "echo> $line\n";
}
    \end{lstlisting}
    \end{itemize}
  \end{block}

\end{frame}

\begin{frame}[fragile]
  \frametitle{Entrées et sorties}

  \begin{block}{L'opérateur $<>$}
    \begin{itemize}
    \item il est possible de considérer une liste de fichiers mis en
      paramètre d'un script comme l'entrée standard
      \begin{lstlisting}{language=bash}
$ perl script.pl file1 file2 file3
      \end{lstlisting}
      \begin{lstlisting}{language=perl}
while( <> )
{
  print "echo> $_";
 }
      \end{lstlisting}
    \item le contenu des fichiers file1, file2 et file3 sont envoyés vers la
      sortie standard
    \end{itemize}
  \end{block}

\end{frame}

\begin{frame}[fragile]
  \frametitle{Entrées et sorties}

  \begin{exampleblock}{Lecture de fichier}
    \begin{itemize}
    \item dans un premier temps, il faut ouvrir le fichier via la fonction
      \textit{open} qui retourne un \textit{handler} sur le fichier (FILE)
      \begin{lstlisting}{language=perl}
open FILE, "filename";
      \end{lstlisting}
    \item la lecture est totalement identique à la lecture sur l'entrée
      standard
      \begin{lstlisting}{language=perl}
while( <FILE> )
{
  print "echo> $_";
}
      \end{lstlisting}
    \end{itemize}
  \end{exampleblock}

  \begin{alertblock}{Test d'ouverture}
    L'ouverture d'un fichier peut échouer, dans ce cas, on peut associer une
    sortie d'erreur avec la fonction \textit{die}
    \begin{lstlisting}{language=perl}
open FILE, "filename" or die "Could not open file: $! \n";
    \end{lstlisting}
    \$! est une variable spéciale contenant la dernière erreur
  \end{alertblock}

\end{frame}

\begin{frame}[fragile]
  \frametitle{Entrées et sorties}

  \begin{block}{Ecriture de fichier}
    \begin{itemize}
    \item comme pour la lecture, il faut ouvrir le fichier via la fonction
      \textit{open} qui retourne un \textit{handler} sur le fichier (FILE)
    \item la seule différence réside dans le contenu de la chaîne de
      caractères contenant le nom du fichier
      \begin{lstlisting}{language=perl}
open FILE, "> filename";
open FILE, ">> filename";
      \end{lstlisting}
    \item $>$ ouvre le fichier en création
    \item $>>$ ouvre le fichier en ajout à la fin
    \item l'écriture est très proche de l'écriture sur la sortie
      standard, on mentionne juste le \textit{handler}
      \begin{lstlisting}{language=perl}
print FILE "Hello", " world", "!\n";
printf FILE "%s n'existe pas \n", $file;
      \end{lstlisting}
    \end{itemize}
  \end{block}

  \begin{alertblock}{Fermeture}
    Après utilisation d'un fichier, il faut le "fermer" : \textit{close FILE;}
  \end{alertblock}

\end{frame}

\begin{frame}[fragile]
  \frametitle{Entrées et sorties}

  \begin{block}{Perl et les \textit{pipes}}
    \begin{itemize}
    \item les entrées-sorties peuvent se faire via des \textit{pipes} entre
      processus
    \item l'entrée d'un processus est la sortie d'un autre processus
    \item Perl permet de se "connecter" en entrée ou en sortie à un autre
      processus via un \textit{pipe}
    \item la philosophie est toujours la même : la chaîne de caractères passée
      en paramètre de la fonction \textit{open} va contenir le symbole | et
      le processus auquel on se connecte
    \end{itemize}
  \end{block}

  \begin{exampleblock}{Exemples}
    \begin{itemize}
    \item en sortie :
      \begin{lstlisting}{language=perl}
open GREP, '| grep -e "s"';
      \end{lstlisting}
    \item en entrée :
      \begin{lstlisting}{language=perl}
open NETSTAT, "netstat |";
while( <NETSTAT> )
{
  # un traitement
}
      \end{lstlisting}
    \end{itemize}
  \end{exampleblock}

\end{frame}

\begin{frame}[fragile]
  \frametitle{Entrées et sorties}

  \begin{block}{\textit{backquote}}
    \begin{itemize}
    \item en \textit{shell}, les \textit{backquotes} permettent d'invoquer
      une commande à l'intérieur d'une autre commande
    \item le retour de la commande entre \textit{backquotes} remplace la
      commande est l'autre commande est exécutée
      \begin{lstlisting}{language=bash}
$ cc -o exe `pkg-config --libs --cflags gtkmm-2.4` test.cc
      \end{lstlisting}
    \item en Perl, on peut exécuter une commande externe et stocker le résultat
      dans une variable pour un traitement ultérieur
    \end{itemize}
  \end{block}

  \begin{exampleblock}{Exemples}
    Le résultat de la commande \textit{ls} est placé dans une variable scalaire
    ou une liste
    \begin{lstlisting}{language=perl}
$lines = `ls -l`;
@lines = `ls -l`;
    \end{lstlisting}
  \end{exampleblock}

\end{frame}

