\section{Les expressions régulières}

\begin{frame}[fragile]
  \frametitle{Les expressions régulières}

  \begin{block}{Définition}
    \begin{itemize}
    \item une expression régulière est un "patron" (\textit{pattern})
      décrivant une part d'une chaîne de caractères
    \item une chaîne de caractères correspond au \textit{pattern} ou non
    \item le \textit{pattern} peut être détecté n'importe où dans la
      chaîne de caractères
    \end{itemize}
  \end{block}

  \begin{exampleblock}{Patterns simples}
    Le \textit{pattern} le plus simple est un caractère ou une suite
    de caractères.
  \end{exampleblock}

  \begin{alertblock}{"Peu importe"}
    Le point (.) permet être remplacé par n'importe quel caractère sauf
    le \textit{newline}.
  \end{alertblock}

\end{frame}

\begin{frame}[fragile]
  \frametitle{Les expressions régulières}

  \begin{block}{Les ensembles}
    \begin{itemize}
    \item au lieu d'utiliser le point, il est possible de définir des
      ensembles
    \item deux possibilités :
      \begin{itemize}
      \item l'opérateur [] : l'un des caractères place entre crochet peut
        apparaître (par exemple, \texttt{[abc]}) ; le symbole "-" permet
        de simplifier les énumérations
      \item des raccourcis : l'ensemble est défini par un \textbackslash et une
        lettre, et désigne un ensemble prédéfini (par exemple, l'ensemble des
        chiffres)
      \end{itemize}
    \end{itemize}
  \end{block}

  \begin{exampleblock}{Exemples}
    \begin{lstlisting}{language=perl}
a[bcd]e	  # abe ou ace ou ade
a[b-y]z	  # abz ou ... ou ayz
a[\t ]b	  # a\tb ou a b
a[0-9]    # a0 ou ... ou a9
    \end{lstlisting}
  \end{exampleblock}

  \begin{alertblock}{Raccourcis}
    \begin{lstlisting}{language=perl}
\d        # idem a [0-9]
\w        # idem a [a-zA-Z0-9_]
\s        # idem a [\t\f\n\r ]
    \end{lstlisting}
  \end{alertblock}

\end{frame}

\begin{frame}[fragile]
  \frametitle{Les expressions régulières}

  \begin{block}{Les "ancres" (\textit{anchors})}
    Une ancre indique que l'on veut que le pattern soit accroché à un élément
    spécifique :
    \begin{itemize}
    \item le début de la chaîne de caractères (\textasciicircum)
    \item la fin de la chaîne (\$)
    \item un mot (\textbackslash b)
    \end{itemize}
  \end{block}

  \begin{exampleblock}{Exemples}
    \begin{lstlisting}{language=perl}
^abc     # 'abcde' est ok mais pas 'xyzabc'
xyz$     # 'abcxyz' est ok mais pas 'xyzabc'
Hello\b  # 'Hello world' est ok mais pas 'Helloworld'
    \end{lstlisting}
  \end{exampleblock}
%$
\end{frame}

\begin{frame}[fragile]
  \frametitle{Les expressions régulières}

  \begin{block}{Les répétitions}
    Il est possible de mentionner que certains \textit{patterns} se répétent.
    Plusieurs fréquences possibles :
    \begin{itemize}
    \item ? : zéro ou une fois
    \item * : zéro ou plusieurs fois
    \item + : une ou plusieurs fois
    \item \{\$n,\$m\} : de n à m fois
    \item \{\$n,\} : au moins n fois
    \item \{,\$m\} : au plus m fois
    \end{itemize}
  \end{block}

  \begin{exampleblock}{Exemples}
    \begin{lstlisting}{language=perl}
ab?     # a ou ab
ab*     # a ou ab ou abb ou ...
ab+     # ab ou abb ou ...
ab{2,3} # abb ou abbb
    \end{lstlisting}
  \end{exampleblock}

\end{frame}

\begin{frame}[fragile]
  \frametitle{Les expressions régulières}

  \begin{block}{Les alternatives}
    \begin{itemize}
    \item un \textit{pattern} peut se composer de plusieurs choix parmi
      un ensemble
    \item l'opérateur de séparation des alternatives est |
    \item dans certains cas, l'utilisation des paranthèses est indispensable
      pour gérer la priorité entre les opérateurs
    \end{itemize}
  \end{block}

  \begin{exampleblock}{Exemples}
    L'opérateur d'alternatives est moins prioritaires que \textasciicircum et
    \$ :
    \begin{lstlisting}{language=perl}
^a|b|c    # identique a (^a)|b|c
a|b|c$    # identique a a|b|(c$)
^(a|b|c)
(a|b|c)$
    \end{lstlisting}
  \end{exampleblock}
%$
\end{frame}

\begin{frame}[fragile]
  \frametitle{Les expressions régulières}

  \begin{block}{L'opérateur de correspondance (\textit{matching})}
    \begin{itemize}
    \item l'application d'un \textit{pattern} sur une chaîne de caractères
      s'effectue à l'aide de m/REGEX/ où m signifie \textit{match} et REGEX est
      le pattern
    \item par défaut, la recherche de correspondance est réalisée sur la
      variable par défaut \$\_
    \item s'il y a correspondance alors \textit{true} est retournée
    \item on peut chercher la correspondance sur une variable autre que \$\_ à
      l'aide de l'opérateur de \textit{binding} : =\textasciitilde
    \end{itemize}
  \end{block}
\end{frame}

\begin{frame}[fragile]
  \frametitle{Les expressions régulières}

  \begin{exampleblock}{Exemples}
    \begin{lstlisting}{language=perl}
if( m/REGEX/ )
{
  print "It matches!\n";
}

if( $string =~ m/REGEX/ )
{
...
}
    \end{lstlisting}
  \end{exampleblock}
%$

  \begin{alertblock}{Encore plus court !}
    On peut omettre le m devant /REGEX/.
    \begin{lstlisting}{language=perl}
if( /REGEX/ )
{
  print "It matches!\n";
}
    \end{lstlisting}
  \end{alertblock}

\end{frame}

\begin{frame}[fragile]
  \frametitle{Les expressions régulières}

  \begin{block}{\textit{Case sensitive}}
    \begin{itemize}
    \item par défaut, l'opérateur de \textit{matching} recherche la
      correspondance exacte
    \item on peut lever le test majuscule / minuscule en ajoutant i à la fin
      de l'opération de correspondance
    \end{itemize}
  \end{block}

  \begin{exampleblock}{Exemples}
    L'utilisateur peut faire o ou O.
    \begin{lstlisting}{language=perl}
{
  print "Voulez-vous supprimer le fichier ? ";
  $answer = <STDIN>;
  redo unless $answer =~ m/^o/i;
}
    \end{lstlisting}
  \end{exampleblock}

  \begin{alertblock}{Le \textit{pattern} dans une variable}
    Le \textit{pattern} peut être une chaîne de caractère.
    \begin{lstlisting}{language=perl}
$regex = 'root|toto';
if( /$regex/ )    # identique a m/root|toto/
{ ... }
    \end{lstlisting}
  \end{alertblock}

\end{frame}

\begin{frame}[fragile]
  \frametitle{Les expressions régulières}

  \begin{block}{Choix}
    Les sous expressions paranthesées dans lequel il y a des alternatives ou
    des patterns peuvent être récupérées, après l'opération de correspondance
    \begin{lstlisting}{language=perl}
$_ = 'toto est un utilisateur';
m/(t\w+) est un utilisateur/;
$name = $1;
    \end{lstlisting}
    \$1 est une \textit{memory variable} et est égal à 'toto'
  \end{block}
%$
  \begin{alertblock}{Durée de vie des variables}
    Les variables \$x issues des opérateurs de \textit{matching} ne sont pas
    touchées jusqu'à la prochaine opération de \textit{matching}
    \textbf{réussie}.
  \end{alertblock}

\end{frame}

\begin{frame}[fragile]
  \frametitle{Les expressions régulières}

  \begin{block}{Substitution}
    \begin{itemize}
    \item les expressions régulières sont aussi utilisées pour remplacer
      une partie d'une chaîne de caractères par une autre chaîne
    \item le \textit{pattern} permet de définir la partie de la chaîne
      qui sera remplacé
    \end{itemize}
    \begin{lstlisting}{language=perl}
s/REGEX/REMPLACEMENT/
    \end{lstlisting}
  \end{block}

  \begin{exampleblock}{Exemples}
    \begin{lstlisting}{language=perl}
$_ = 'user: root';
s/user/utilisateur/;      # $_ = 'utilisateur: root'

$name = 'user: toto';
$name =~ s/toto/titi/;    # $name = 'user: titi'
    \end{lstlisting}
  \end{exampleblock}
%$
  \begin{alertblock}{Retour d'une substitution}
    Une substitution retourne \textit{true} si elle a eu lieue.
  \end{alertblock}

\end{frame}

\begin{frame}[fragile]
  \frametitle{Les expressions régulières}

  \begin{block}{Changement de comportement}
    On peut ajouter un modificateur de comportement :
    \begin{itemize}
    \item i : suppression du mode "case sensitive"
    \item g : substitue toutes les occurrences du \textit{pattern}
    \item s : change le comportement du point $\rightarrow$ les
      \textit{newlines} sont pris en compte
    \end{itemize}
    \begin{lstlisting}{language=perl}
s/REGEX/REMPLACEMENT/
    \end{lstlisting}
  \end{block}

  \begin{exampleblock}{Exemples}
    \begin{lstlisting}{language=perl}
$_ = 'toto tata toto titi';
s/toto/titi/g;               # $_ = 'titi tata titi titi';

$_ = 'toto tata Toto titi';
s/toto/titi/i;               # $_ = 'titi tata titi titi';

s/\/\*.*\*\///s;             # supprimer les commentaires d'un code source C
    \end{lstlisting}
  \end{exampleblock}

\end{frame}

\begin{frame}[fragile]
  \frametitle{Les expressions régulières}

  \begin{block}{\textit{Memory variable}}
    Lors de la substitution, les variables mémoires sont affectées et donc
    réutilisable dans la chaîne de remplacement.
    \begin{lstlisting}{language=perl}
$name = 'user: toto [tutu]';
$name =~ s/toto (\[tutu\]*)/$1 titi/;
    \end{lstlisting}
  \end{block}
%$

  \begin{alertblock}{Autres délimiteurs}
    \begin{itemize}
    \item dans certains cas, les délimiteurs / / sont génants et nécessitent
      l'utilisateur de \textbackslash
    \item on peut les changer par \#...\# ou (...) ou <...> ou \{...\} ou [...]
    \end{itemize}
    \begin{lstlisting}{language=perl}
m/\/usr\/bin\/perl/
m#/usr/bin/perl#
s#/usr/bin/perl#/bin/perl#
    \end{lstlisting}
  \end{alertblock}

\end{frame}

