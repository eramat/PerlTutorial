\section{Les modules}

\begin{frame}[fragile]
  \frametitle{Les modules}

  \begin{block}{Définition}
    \begin{itemize}
    \item Perl dispose d'une multitude de bibliothèques et de modules
    \item il existe 2 façons d'utiliser un module :
      \begin{itemize}
      \item le code du module est intégré au code (comme s'il avait été écris
        dans le code) avec l'instruction \textit{require}
      \item le code du module reste dans l'espace de nommage du module avec
        l'instruction \textit{use}
      \item avec \textit{use}, il faut utiliser \textit{import} si l'on veut
        obtenir le même comportement
      \end{itemize}
    \end{itemize}
  \end{block}

  \begin{exampleblock}{Exemples}
    Une fonction \textit{search} est définie dans le module \textit{utils}.
    \begin{lstlisting}[language=perl]
require "utils.pl";
search($string, $value);

use utils;
utils::search($string, $value);
    \end{lstlisting}
  \end{exampleblock}

\end{frame}
