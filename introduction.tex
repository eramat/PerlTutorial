\section{Introduction}

\begin{frame}
  \frametitle{Introduction}

  \begin{block}{Histoire}
    \begin{itemize}
      \item la première \textit{release} de Perl date de 1987 (avant la
        naissance du Web !)
      \item le livre de référence ``Programming perl'' est publié en ??? et
        Perl 4.0 est disponible
      \item Perl 5.0, la version actuelle, apparaît en 1994 :
        \begin{itemize}
        \item nouvelle conception extensible
        \item introduction des modules tiers
	\item possibilité de définir des structures de données complexes
        \item des capacités orientées objets
        \end{itemize}
    \end{itemize}
  \end{block}

  \begin{exampleblock}{Son nom}
    \begin{itemize}
      \item \textit{Practical Extraction and Reporting Language}
      \item Perl est le langage
      \item perl est le compilateur
      \item on n'écrit jamais PERL
    \end{itemize}
  \end{exampleblock}

\end{frame}

\begin{frame}
  \frametitle{Introduction}

  \begin{block}{Caractéristiques}
    \begin{itemize}
      \item un langage de programmation général et de haut niveau
      \item libre et open source
      \item intégre les meilleurs choses du C, de \textit{awk}, de \textit{sed}
        et de d'autres langages
      \item un mélange d'orienté objets et de procédural
      \item rapide
      \item flexible
      \item sûr
      \item \ldots et sympa !
    \end{itemize}
  \end{block}

  \begin{exampleblock}{Utilisations}
    \begin{itemize}
      \item traitement sur des textes et des chaînes de caractères
      \item tâches d'administration système et réseau
      \item programmation web et CGI
      \item interaction avec des bases de données
    \end{itemize}
  \end{exampleblock}

\end{frame}

\begin{frame}
  \frametitle{Introduction}

  \begin{block}{Pourquoi utiliser Perl ?}
    \begin{itemize}
      \item prototypage rapide
      \item compact donc rapide à écrire et simple à debugger
      \item la gestion mémoire est automatique,
      \item les tableaux, les listes, les tables de hachage sont natifs
      \item portable : Unix, Windows, Mac OS X
      \item très expressif
    \end{itemize}
  \end{block}

  \begin{exampleblock}{Philosophie}
    \begin{itemize}
      \item ``les choses simples doivent être simples et les choses difficiles
        doivent être possibles''
      \item une des devises de Perl est : \textit{there is more than one way
        to do it} $\rightarrow$ ``il y a plus d'une façon de le faire''
    \end{itemize}
  \end{exampleblock}

\end{frame}

\begin{frame}[fragile]
  \frametitle{Introduction}

  \begin{block}{Un programme Perl}
    \begin{itemize}
      \item un simple fichier texte (écrit avec vi ou emacs !)
      \item la syntaxe est proche du C :
        \begin{itemize}
        \item les espaces ne jouent pas de rôle particulier
        \item les instructions se terminent par des points virgules
        \end{itemize}
      \item les commandes sont de type ``fin de ligne'' et commencent
        par le caractère \#
      \item les variables ne doivent pas être déclarées
      \item l'interpréteur Perl compile et exécute les scripts
    \end{itemize}
  \end{block}

\end{frame}

\begin{frame}[fragile]
  \frametitle{Introduction}

  \begin{exampleblock}{Execution}
    Deux possibilités pour exécuter un script Perl
    \begin{itemize}
      \item en ligne de commmande :
        \begin{lstlisting}[language=bash]
$ perl script.pl
          \end{lstlisting}
      \item à partir d'un fichier script exécutable :
        \begin{itemize}
        \item ajout d'une ligne en début de ligne pour indiquer la commande
          à exécuter pour exécuter le script (\textit{shebang line})
          \begin{lstlisting}[language=bash]
#!/usr/bin/perl
          \end{lstlisting}
        \item fichier en mode exécutable :
          \begin{lstlisting}[language=bash]
$ chmod +x script.pl
          \end{lstlisting}
        \item exécution :
          \begin{lstlisting}[language=bash]
$ ./script.pl
          \end{lstlisting}
        \end{itemize}
    \end{itemize}
  \end{exampleblock}

\end{frame}

\begin{frame}[fragile]
  \frametitle{Introduction}

  \begin{alertblock}{Options}
    Il existe plusieurs options d'exécution dont le mode ``avec warnings'' :
    \begin{lstlisting}[language=bash]
$ perl -w script.pl
    \end{lstlisting}
  \end{alertblock}

  \begin{exampleblock}{Mon premier script Perl}
    \begin{lstlisting}[language=perl]
#!/usr/bin/perl

print "Hello world\n";
    \end{lstlisting}
  \end{exampleblock}

\end{frame}

