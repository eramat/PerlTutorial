\documentclass[11pt,a4paper,oneside]{article}
\usepackage{listings}
%\usepackage{ucs}
\usepackage[latin1]{inputenc}
\usepackage[T1]{fontenc}
\usepackage{txfonts}
\usepackage{lmodern}
\usepackage[pdftex]{thumbpdf}
\usepackage[citecolor={purple},linkcolor={blue},urlcolor={blue},
   a4paper,colorlinks,breaklinks]{hyperref}
\usepackage{txfonts}
\usepackage{hyperref}
\usepackage{xcolor}
\usepackage{graphicx}

\graphicspath{{../figures/}}

\renewcommand\familydefault{\sfdefault}

\usepackage{vmargin}
\setmarginsrb{3.3cm}{2cm}{1cm}{1cm}{1cm}{1cm}{0.5cm}{0.5cm}

%% \usepackage{fullpage}
\usepackage{color}
\usepackage{url}
\usepackage[french]{babel}
\usepackage{listings}

\author{Eric Ramat\\\url{ramat@lisic.univ-littoral.fr}}
\title{\textbf{\textbf{TP Perl}}\\
\emph{Scripts syst�me}}
\newcommand{\orangeline}{\rule{\linewidth}{1mm}}

\lstset{language=C++,extendedchars=true,inputencoding=latin1,
    basicstyle=\ttfamily\small, commentstyle=\ttfamily\color{red},
      showstringspaces=false,basicstyle=\ttfamily\small}


\newcommand{\background}{
\setlength{\unitlength}{1in}
\begin{picture}(0,0)
 \put(-1.4,-7.45){\includegraphics[height=28.7cm]{background1}}
\end{picture}}

\begin{document}
\maketitle
\background

\begin{flushright}
  Dur�e : 9 heures\end{flushright}

\noindent\orangeline

L'objectif de ce TP est de comprendre l'�criture de scripts syst�me en Perl.

\section{Travail}

\textbf{Exercice 1.} Changer de mani�re r�cursive les droits des fichiers de
type texte (extension .txt) sur une arboresence. On indiquera au script le
r�pertoire d'o� doit commencer le changement.\\

Voici un exemple d'appel :
\begin{lstlisting}[language=bash]
$ ./change_permissions.pl mon_repertoire
\end{lstlisting}

\section{Correction}

\textbf{Exercice 1.}

\begin{lstlisting}[language=bash]
print "Changing permissions on $ARGV[0]\n";

chmod 0755,$ARGV[0];

$whereami=`pwd`;
chomp($whereami);
push @dir,$whereami."/".$ARGV[0];

while ($dirs=pop @dir) {

	$whereami=`pwd`;
	chomp($whereami);
	chomp($dirs);
	chdir $dirs;
	print "changing to $dirs\n";


	while (<*>) {

		next if ($_ eq ".");
		next if ($_ eq "..");

		if (-d $_) {
			next if (-l $_);
			chmod 0755,$_;
			push @dir,$dirs."/".$_;
			#print "pushing ".$dirs."/".$_."\n";
		} else {
			if (/.*\.cgi/) {
				chmod 0755,$_;
			} else {
				chmod 0744,$_;
			}
		}
	}
	chdir $whereami;
	#print "changing to $dirs\n";

}
\end{lstlisting}


\end{document}
